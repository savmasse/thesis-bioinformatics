\chapter{Introduction}

In this thesis we study the use of variational Bayesian inference (VI) in phylogenetic modelling. We will attempt to show that variational algorithms can compete with the Markov chain Monte Carlo (MCMC) methods that are traditionally used. Specifically, we will apply VI to the state-of-the-art phylogenetic models Whale and Beluga that are currently being developed at the VIB by my supervisor Arthur Zwaenepoel. There is currently an unfortunate limit to the size of the analyses that can be performed because of the nature of MCMC. Typically only species trees of around 9-12 species can be analysed within a reasonable time frame. In this thesis we will demonstrate that variational algorithms could be a viable alternative to MCMC methods. VI could greatly speed up the process, allowing for the analysis of much larger models with more species. 
\\
\\
To that end, this thesis is structured as follows: first a literature study is presented, reviewing the basic concepts of Baysian inference, the general idea behind the variational approach and explaining common variational algorithms. Next we will present the experiments performed during this project and discuss the results. Finally the methods and data used in the experiments are described in the Methods and Materials chapter.