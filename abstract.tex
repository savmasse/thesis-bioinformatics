
\begin{abstract}
Variational inference (VI) algorithms are becoming increasingly common for the Bayesian inference, as they can be significantly faster than traditional Markov chain Monte Carlo (MCMC) inference methods. VI methods are already in use in the field of bioinformatics, but often algorithms have to be manually derived for each different probability model. This can be impractical in the development stage when models are constantly still changing.
\\
In this thesis an overview is given of common variational inference algorithms that can be applied to any model. 
Several experiments were performed on state-of-the-art phylogenetic models currently being developed at the VIB. 
We thus show that generic VI methods can be easily applied to existing phylogenetic models, providing fast approximate solutions to larger problems than can traditionally be tackled with MCMC. 
\end{abstract}

\begin{abstract}
Variationele algoritmen (VI) worden alsmaar meer gebruikt voor Bayesiaanse inferentie, en kunnen significant sneller zijn dan de traditionele Markov chain Monte Carlo (MCMC) methoden. VI wordt reeds toegepast in de bioinformatica, maar vaak moeten deze algoritmen manueel worden afgeleid voor elk verschillend probabilistisch model. Dit kan onpraktisch zijn in het ontwikkelingsstadium wanneer het model nog constant aangepast wordt. 
\\
In deze thesis wordt een overzicht gegeven van generieke variationele algoritmen die op elek statistisch model kunnen worden toegepast. We voerden verschillende experimenten uit op state-of-the-art phylogenetische modellen die in het VIB werden ontwikkeld. Zodoende tonen we aan dat VI direct toepasbaar is op bestaande modellen, en snelle oplossingen levert voor grotere problemen dan men met traditionele MCMC kan aanpakken.

\end{abstract}